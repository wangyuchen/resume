\documentclass[11pt]{article}

\usepackage[letterpaper, margin=.5in]{geometry}
\usepackage{mdwlist}
\usepackage{hyperref}
\usepackage{enumitem}
\usepackage{fontspec}
\usepackage[compact]{titlesec}

\pagestyle{empty}
\setlength{\skip\footins}{.3em}
\setmainfont{Times New Roman}



% format two pieces of text, one left aligned and one right aligned
\newcommand{\headerrow}[2]
{\begin{tabular*}{\linewidth}{l@{\extracolsep{\fill}}r}
	#1 & #2 \\
\end{tabular*}
}

% make "C++" look pretty when used in text by touching up the plus signs
\newcommand{\CPP}
{C\nolinebreak[4]\hspace{-.05em}\raisebox{.22ex}{\footnotesize\bf ++}}

% enumitem setup
\setlist[description]{nolistsep, leftmargin=\parindent,
											labelindent=.33\parindent, rightmargin = .33\parindent}
\setlist[itemize]{nolistsep, rightmargin = .33\parindent}

% and the actual content starts here
\begin{document}

\begin{center}
	\huge\textbf{YUCHEN WANG}
\end{center}

\noindent\headerrow{+1 (573) 239-5737}{2601 Old Highway 63 South}
\headerrow
{\href{mailto:yuchen.wang@mail.missouri.edu}{yuchen.wang@mail.missouri.edu}}
{Columbia, Missouri, 65201}

\vspace{-.8em}
\hrule

\subsubsection*{\centering EDUCATION}
\vspace{-.3em}

\headerrow
	{\textbf{\emph{Master of Arts} in Statistics}, University of Missouri, Columbia, Missouri}
	{\textbf{expected May 2015}}
	\begin{itemize}
		\item Maintained 4.0 cumulative GPA with A+ in Probability Theory, Linear Models and Bayesian Analysis.
	\end{itemize}

\noindent\headerrow
	{\textbf{\emph{Bachelor of Science} in Statistics}, East China Normal University, Shanghai, China}
	{\textbf{June 2013}}
	\begin{itemize}
		\item Took \emph{Coursera} courses CS50, machine learning, high performance computing, etc.
		\item Nominated for best thesis: \textit{Residual Control Charts based on semi-parametric regression models}
	\end{itemize}

\noindent\headerrow
	{\textbf{\emph{Minor} in Computer Science}, Shanghai Jiao Tong University, Shanghai, China}
	{\textbf{June 2013}}
	\begin{itemize}
		\item Completed all undergraduate curriculum of Computer Science and a graduation thesis.
	\end{itemize}


\vspace{0.2em}
\hrule


\subsubsection*{\centering RELEVENT EXPERIENCE}
\vspace{-.3em}

\headerrow
{\textbf{Research Assistant}, University of Missouri, Columbia, Missouri}
{\textbf{August 2014 – Present}}
\begin{itemize}
	\item The only research assistant in our graduate class; Collaborated with PhD students in different research projects.
	\item Optimized a bayesian MCMC algorithm in Fortran for gene data. Reduced computation time from more than 3 hours to less than 20 seconds.
	\item Parallelized a large-scale Bayes factor computation program in Fortran using both \emph{OpenMP} and \emph{MPI}; Tuned for distributed computing on different kinds of high performance computing clusters.
\end{itemize}

\noindent\headerrow
{\textbf{Research Assistant}, East China Normal University, Shanghai, China}
{\textbf{July 2012 - June 2014}}
\begin{itemize}
	\item Joined a research team of Statistical Process Control (SPC) in tobacco manufacturing as a part-time research assistant.
	\begin{itemize}
		\item The only undergraduate student in a research team of 7 members consisting graduate students and faculties.
		\item Experienced in analyzing real-world big data (GBs per hour) from observational studies on Amazon EC2 using parallelized algorithms.
		\item Designed a control scheme for controlling tobacco dehumidification process, which is the most critical and complicated control unit among all tobacco manufacturing processes.
		\item Collaborated with both quality control team at the tobacco manufacturing factory and software engineering team of an IT company. The final quality control product could precisely detect abnormalities of the manufacturing in advance so as to minimize the cost of failure products.
	\end{itemize}

	\item Hired as a full-time research assistant in charge of the SPC project after graduation. Led a team of 21 members.
	\begin{itemize}
		\item Mentored 7 members in the team. Trained all new members about previously used models and programs.
		\item Developed an R package for the project containing raw data, data cleaning routines, statistical algorithms and visualization tools. Programmed over 2,500 lines of R code for this project.
	\end{itemize}

	\item Recruited 3 undergraduate students for the development of a statistical software with user interface as a product for a two stage randomized experiment project.
	\begin{itemize}
		\item Worked remotely with the rest of the team while studying abroad.
		\item Designed and developed an interactive data visualization web application in R and \emph{Shiny}.
	\end{itemize}
\end{itemize}


	\vspace{0.2em}
	\hrule

	\subsubsection*{\centering RELATED SKILLS}
	\vspace{-.3em}

	\begin{description}
		\item[Statistical Computing] Highly skilled in R; Developer of several R packages (see my GitHub: \href{https://www.github.com/wangyuchen}{@wangyuchen}); Mostly used packages including \emph{MCMCpack}, \emph{R2OpenBUGS} and \emph{coda} for Bayesian computing; \emph{spatstat} and \emph{splancs} for spatial point pattern analysis; \emph{ggplot2} and \emph{knitr} for graphics and reports. Proficent in SAS procedures including \emph{glm}, \emph{nlin}, \emph{genmod} and \emph{mixed}.
		\item[General Purpose Programming] Solid understanding of object-oriented programming with \emph{\CPP}. Extensive parallel computing experience with Fortran (\emph{OpenMP/MPI}). Experience with \emph{Python} and \emph{Java}.
		\item[Specialities in Statistics] Bayesian analysis; Spatial statistics; Machine Learning and Data Mining; Semi-parametric regression models for longitudinal data and Statistical Process Control.
		\item[Mathematical Modeling] Proficient in \emph{MATLAB} through 3 Mathematical Contest in Modeling experience. Familiar with operations research techniques in \emph{LINGO}.
		\item[Operating Systems] Effective in Unix/Linux environment and Shell scripts. Responsible for HPC cluster management at ECNU. Experience in parallel programming on Linux clusters.
	\end{description}


\let\thefootnote\relax\footnote{Last update on \today}

\end{document}
