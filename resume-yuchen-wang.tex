% resume.tex
% vim:set ft=tex spell:

\documentclass[10pt]{article}
\usepackage[letterpaper, margin=.5in]{geometry}
\usepackage{mdwlist}
\usepackage{hyperref}
\usepackage{enumitem}

\usepackage{fontspec}
\setmainfont{Times New Roman}

% \usepackage[T1]{fontenc}
% \usepackage{textcomp}
% \usepackage{tgpagella}
\pagestyle{empty}
% \setlength{\tabcolsep}{0em}



% indentsection style, used for sections that aren't already in lists
% that need indentation to the level of all text in the document
\newenvironment{indentsection}[1]%
{\begin{list}{}%
	{\setlength{\leftmargin}{#1}}%
	\item[]%
}
{\end{list}}

% opposite of above; bump a section back toward the left margin
\newenvironment{unindentsection}[1]%
{\begin{list}{}%
	{\setlength{\leftmargin}{-0.5#1}}%
	\item[]%
}
{\end{list}}

% format two pieces of text, one left aligned and one right aligned
\newcommand{\headerrow}[2]
{\begin{tabular*}{\linewidth}{l@{\extracolsep{\fill}}r}
	#1 &
	#2 \\
\end{tabular*}}

% make "C++" look pretty when used in text by touching up the plus signs
\newcommand{\CPP}
{C\nolinebreak[4]\hspace{-.05em}\raisebox{.22ex}{\footnotesize\bf ++}}




% and the actual content starts here
\begin{document}

\begin{center}
\huge \textbf{YUCHEN WANG\\}
\end{center}

\noindent\headerrow
{(573) 239-5737}
{2601 Old Highway 63 South}
\\
\headerrow
{\href{mailto://yuchen.wang@mail.missouri.edu}{yuchen.wang@mail.missouri.edu}}
{Columbia, Missouri, 65201}

\vspace{0.2em}
\hrule
\vspace{-1em}


\subsection*{\centering EDUCATION}

\headerrow
	{\textbf{Master of Arts in Statistics}}
	{\textbf{expected June 2015}}
\\
\headerrow
	{Department of Statistics, University of Missouri-Columbia}
	{Missouri, United States}
\\
\headerrow
	{\textbf{Bachelor of Science in Statistics}}
	{\textbf{June 2013}}
\\
\headerrow
	{Department of Statistics and Actuarial Science, East China Normal University}
	{Shanghai, China}
\\
\headerrow
	{\textbf{Minor (36 credits) in Computer Technology and Application}}
	{\textbf{June 2013}}
\\
\headerrow
	{Department of Computer Science and Engineering, Shanghai Jiao Tong University}
	{Shanghai, China}


\vspace{0.2em}
\hrule
\vspace{-1em}



\subsection*{\centering RELEVENT EXPERIENCE}
\headerrow
	{\textbf{Research Assistant, East China Normal University, Shanghai}}
	{\textbf{July 2012 – December 2013}}
\begin{itemize}[topsep=2pt, itemsep=-2pt, nolistsep]
	\item To apply my statistics knowledge and programming skills into practice, I persuaded professor Xiaolong Pu to let me participate in his project as a student research assistant about statistical process control in tobacco processing.
	\begin{itemize}[topsep=2pt, itemsep=-2pt]
		\item The only undergraduate student in a research team of 7 consists of graduate students and faculties.
		\item Designed the control plan for tobacco dehumidification process under guidance, which is the most critical and complicated control unit among all tobacco processing processes.
		\item Two papers were completed (unpublished): one is on semi-parametric regression model for clustered data and the other is on design and implementation of a quality control software.
	\end{itemize}
	\item With the project moved into a new phase, the team expanded to 21 members. After graduation, I was hired as a full-time research assistant and became a team leader.
	\begin{itemize}[topsep=2pt, itemsep=-2pt]
		\item Instructed new members to learn about previously used models and programs. Tutored 7 members once a week to re-write previous programs in a unified manner.
		\item Developed an R package based on those re-written programs, which contains raw data, data cleaning methods, statistical algorithms and visualization methods used in the project.
		\item A paper discussing ridge estimators in models with correlated covariates is in review.
	\end{itemize}
\end{itemize}

\noindent\headerrow
	{\textbf{Cluster Administrator, East China Normal University, Shanghai}}
	{\textbf{March 2013 – December 2013}}
\begin{itemize}[topsep=2pt, itemsep=-2pt]
    \setlength{\itemsep}{-2pt}
	\item Built the high-performance computing cluster for School of Finance and Statistics, East China Normal University, which provides web-based cluster computing environment of R and MATLAB to graduate students.
	\item Responsible for cluster configuration, maintenance, access management and manual writing.
\end{itemize}

\noindent\headerrow
	{\textbf{Keynote lecture: \emph{R and Object-oriented Statistical Analysis}, $5^{th}$ China-R Conference, Shanghai}}
	{\textbf{November 4, 2012}}
\begin{itemize}[topsep=2pt, itemsep=-2pt]
    \setlength{\itemsep}{-2pt}
	\item Recommended by professor Yincai Tang to give a 45 minutes lecture about object systems in R.
	\item Introduced the usage of S3, S4 and R5 objects in R.
	\item Explained how these object systems can be utilized to make analysis more simple and accurate.
\end{itemize}

\noindent\headerrow
	{\textbf{Operations and Maintenance Intern, Digital Resource, Shanghai}}
	{\textbf{January 2011 – February 2011}}
\begin{itemize}[topsep=2pt, itemsep=-2pt]
    \setlength{\itemsep}{-2pt}
	\item Responsible for operation and maintenance for Shanghai Testing Center's information system.
	\item Participated in the construction of information infrastructure for Advanced Financial School of Shanghai Jiao Tong University.
	\item Experienced and became interested in computer networks.
\end{itemize}

\noindent\headerrow
	{\textbf{Analyst Intern, Shanghai Association for Quality, Shanghai}}
	{\textbf{July 2010 – August 2010}}
\begin{itemize}[topsep=2pt, itemsep=-2pt]
	\item Conducted the customer satisfaction survey sampling for the Xujiahui business circle. Interviewed over 1,000 customers.
	\item Responsible for street interview, data entry, analysis and reporting writing. Experienced the whole cycle of survey sampling.
\end{itemize}

\vspace{0.2em}
\hrule
\vspace{-1em}

\subsection*{\centering SKILLS}

\begin{description*}
	\item[Statistical Computing] Highly skilled in R; Developer of several R packages (see my GitHub @wangyuchen); Proficient in OpenBUGS for Bayesian computing; Profient in SAS; Familiar with SPSS.
	\item[General Purpose Programming] Solid understanding of object-oriented programming with C++; Extensive scientific computing experience with Fortran; Experience with Java.
	\item[Specialities in Statistics] Bayesian analysis; Spatial statistics; Machine Learning and Data Mining; Semi-parametric regression models for longitudinal data and Statistical Process Control.
	\item[Mathematical Modeling] Profient in MATLAB through experience in Mathematical Modeling Contest for 3 times. Familiar with operations research techniques in LINGO.
	\item[Operating Systems] Effective in Unix/Linux environment; Responsible for HPC cluster managerment at ECNU; Experience in parallel programming on linux clusters.
	% \item[Statistical Computing:] Outstanding skills in R. Developer and maintainer of a CRAN package \emph{jtrans}. More repos on my GitHub: \url{http://www.github.com/wangyuchen}. Proficient in SAS and SPSS for standard statistical analysis.
	% \item[General Purpose Programming:] Extensive object-oriented programming experience with \CPP. .
	% \item[Operating Systems:] Strong interest in Unix-like systems, especially in the aspect of server management.
	% \item[Specialities in Statistics:] Semi-parametric models; Longitudinal data; Time series analysis; Bayesian analysis; Nonparametric methods and other computationally intensive statistical methods.
	% \item[Mathematical Modeling:] Adept at MATLAB and LINGO during Mathematical Contest in Modeling. Experience with Octave and Gnuplot during the \emph{Machine Learning} course on Coursera.
	% \item[Others:] Effective in \emph{knitr}, in which I can combine my favorite markup languages (Markdown/\LaTeX) with code/outputs produced by a script language (R).
	% \item[Web:] Web data crawler; Spatial analysis and geocoding; Web-based data visualization using interactive plots; Experience with HTML and CSS; Blog in Jekyll and builder of \url{http://www.biostat.com.cn}.
\end{description*}


\let\thefootnote\relax\footnote{Last update on \today}



% \subsection*{\centering HONORS}
% \begin{unindentsection}{\parindent}
% 	\begin{itemize}
% 		\item
% 		\headerrow
% 			{\emph{Say Goodbye to Copy and Paste in Statistical Analysis}, $1^{st}$ R salon, East China Normal University}
% 			{\textbf{Dec 18, 2013}}
% 		\item
% 		\headerrow
% 			{\emph{R and Object-oriented Statistical Analysis}, $5^{th}$ China-R Conference, Shanghai}
% 			{\textbf{Nov 4, 2012}}
% 		\item
% 		\headerrow
% 			{\textbf{Successful Participant}, Mathematical Contest in Modeling, COMAP}
% 			{\textbf{Mar 2012}}
% 		\item
% 		\headerrow
% 			{\textbf{Third Price}, Mathematical Contest in Modeling, East China Normal University}
% 			{\textbf{Oct 2011}}
%
% 	\end{itemize}
% \end{unindentsection}


\end{document}
