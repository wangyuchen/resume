\documentclass[11pt]{article}

\usepackage[letterpaper, margin=.5in]{geometry}
\usepackage{mdwlist}
\usepackage{hyperref}
\usepackage{enumitem}
\usepackage{fontspec}
\usepackage[compact]{titlesec}

\pagestyle{empty}
\setlength{\skip\footins}{.3em}
\setmainfont{Times New Roman}



% format two pieces of text, one left aligned and one right aligned
\newcommand{\headerrow}[2]
{\begin{tabular*}{\linewidth}{l@{\extracolsep{\fill}}r}
	#1 & #2 \\
\end{tabular*}
}

% make "C++" look pretty when used in text by touching up the plus signs
\newcommand{\CPP}
{C\nolinebreak[4]\hspace{-.05em}\raisebox{.22ex}{\footnotesize\bf ++}}

% enumitem setup
\setlist[description]{nolistsep, leftmargin=\parindent,
											labelindent=.33\parindent, rightmargin = .33\parindent}
\setlist[itemize]{nolistsep, rightmargin = .33\parindent}

% and the actual content starts here
\begin{document}

\begin{center}
	\huge\textbf{YUCHEN WANG}
\end{center}

\noindent\headerrow{+1 (573) 239-5737}{2601 Old Highway 63 South}
\headerrow
{\href{mailto:yuchen.wang@mail.missouri.edu}{yuchen.wang@mail.missouri.edu}}
{Columbia, Missouri, 65201}

\vspace{-.8em}
\hrule


\subsubsection*{\centering EDUCATION}
\vspace{-.4em}

\headerrow
	{\textbf{Master of Arts in Statistics}, University of Missouri, Columbia, Missouri}
	{\textbf{expected May 2015}}
	\begin{itemize}
		\item GPA 4.0 with A+ in Probability Theory, Linear Models and Bayesian Analysis.
	\end{itemize}

\noindent\headerrow
	{\textbf{Bachelor of Science in Statistics}, East China Normal University, Shanghai, China}
	{\textbf{June 2013}}
	\begin{itemize}
		\item Took Coursera courses CS50, machine learning, high performance computing, etc.
		\item Nominated for best thesis: \textit{Residual Control Charts based on semi-parametric regression models}
	\end{itemize}

\noindent\headerrow
	{\textbf{Minor in Computer Science}, Shanghai Jiao Tong University, Shanghai, China}
	{\textbf{June 2013}}
	\begin{itemize}
		\item -- Covered all undergraduate curriculum of Computer Science. Thesis required.
	\end{itemize}

\vspace{0.2em}
\hrule


\subsubsection*{\centering RELEVENT EXPERIENCE}
\vspace{-.4em}

\headerrow
{\textbf{Research Assistant}, University of Missouri, Columbia, Missouri}
{\textbf{August 2014 – Present}}
\begin{itemize}
	\item The only research assistant in our graduate class; Collaborated with PhD students in different projects.
	\item Optimized a bayesian MCMC algorithm in Fortran for gene data. Reduced computation time from more than 3 hours to less than 20 seconds.
	\item Parallelized a large-scale Bayes factor computation program in Fortran using both OpenMP and MPI; Tuned for dis- tributed computing on different kinds of high performance computing clusters.
\end{itemize}

\noindent\headerrow
{\textbf{Research Assistant}, East China Normal University, Shanghai, China}
{\textbf{July 2012 - June 2014}}
\begin{itemize}
	\item To apply my statistics knowledge and programming skills into practice, I joined a research team of statistical process control (SPC) in tobacco manufacturing as a part-time research assistant.
	\begin{itemize}
		\item The only undergraduate student in a research team of 7 members consisting graduate students and faculties.
		\item Experienced in analyzing read-world data from observational studies with problems such as measurement error and high-frequency data (GBs per hour).
		\item Designed a model for controlling tobacco dehumidification process, which is the most critical and complicated control unit among all tobacco manufacturing processes.
		\item Collaborated with both quality control team at the tobacco manufacturing factory and software engineering team of an IT company. The final quality control product could detect abnormalities of the manufacturing in advance so as to minimize the cost with failure products.
	\end{itemize}

	\item With the SPC project moved into a new phase, the team expanded to 21 members. In the mean time, I was hired as a full-time research assistant after graduation and became the team leader.
	\begin{itemize}
		\item Mentored 7 members in the team. Trained all new members about previously used models and programs.
		\item  Developed an R package for the project containing raw data, data cleaning routines, statistical algorithms and visualization tools. Programmed over 2,500 lines of R code for this project.
	\end{itemize}

	\item To develop a GUI software as a product for a two stage randomized experiment project, I recruited 3 undergraduate students into the development of a statistical software with user interface.
	\begin{itemize}
		\item Worked remotely with the rest of the team while I was studying abroad.
		\item Developed the statistical algorithm in R; Designed the user interface and data visulization tools in Shiny.
	\end{itemize}
\end{itemize}

\vspace{0.2em}
\hrule

\subsubsection*{\centering RELATED SKILLS}
\vspace{-.4em}

\begin{description}
	\item[Statistical Computing] Highly skilled in R; Developer of several R packages (see my GitHub: \href{https://www.github.com/wangyuchen}{@wangyuchen}); Proficient in R2OpenBUGS for Bayesian computing; Experience in spatstat for spatial point pattern analysis; Profient in SAS.
	\item[General Purpose Programming] Solid understanding of object-oriented programming with C++; Extensive scientific computing experience with Fortran; Experience with Java.
	\item[Specialities in Statistics] Bayesian analysis; Spatial statistics; Machine Learning and Data Mining; Semi-parametric regression models for longitudinal data and Statistical Process Control.
	\item[Mathematical Modeling] Profient in MATLAB through 3 Mathematical Contest i Modeling experience. Familiar with operations research techniques in LINGO.
	\item[Operating Systems] Effective in Unix/Linux environment; Responsible for HPC cluster managerment at ECNU; Experience in parallel programming on linux clusters.
\end{description}

\let\thefootnote\relax\footnote{Last update on \today}

\end{document}
